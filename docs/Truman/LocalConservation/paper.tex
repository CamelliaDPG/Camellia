\documentclass[letterpaper]{article}
\usepackage[margin=1.0in]{geometry}
\usepackage{authblk}

\title{Locally Conservative Discontinuous Petrov-Galerkin Finite Elements for
Fluid Problems}
\author{Truman Ellis} 
\author{Leszek Demkowicz}
\author{Jesse Chan}
\affil{Institute for Computational Engineering and Sciences,\\
The University of Texas at Austin, \\
Austin, TX 78712}
\date{}

\begin{document}
\maketitle

\begin{abstract}
\end{abstract}

\section{Intoduction}
Verteeg and Malalasekera, in \emph{An Introduction to Computational Fluid
Dynamics: The Finite Volume Method}\cite[p. 110-113]{IntroCFD} cite three
characteristics that they consider essential to any numerical discretization
of convection-diffusion type problems: conservativeness, boundedness, and
transportiveness.

Perot\cite{Perot2011} argues
\begin{quote}
Accuracy, stability, and consistency are the mathematical concepts that are
typically used to analyze numerical methods for partial differential equations
(PDEs). These important tools quantify how well the mathematics of a PDE is
represented, but they fail to say anything about how well the physics of the
system is represented by a particular numerical method. In practice, physical
fidelity of a numerical solution can be just as important (perhaps even more
important to a physicist) as these more traditional mathematical concepts. A
numerical solution that violates the underlying physics (destroying mass or
entropy, for example) is in many respects just as flawed as an unstable
solution.
\end{quote}

The discontinuous Petrov-Galerkin finite element method has been described as
least squares finite elements with a twist. The key difference is that least
square methods seek to minimize the residual of the solution in some Hilbert
space norm, while DPG seeks the minimization in a dual norm through the
inverse Riesz map. Exact mass conservation has been an issue that has plagued
least squares finite elements for a long time. Several approaches have been
used to try to adress this. Chang and Nelson\cite{ChangNelson1997} developed
the 'restricted LSFEM'\cite{ChangNelson1997} be augmenting the least squares
equations with a Lagrange multiplier explicitly enforcing mass conservation
element-wise. Our conservative formulation of DPG takes a similar approach and
both methods share similar negative of transforming a minimization method to a
saddle-point problem.

The discontinuous Petrov-Galerkin finite element method has shown a lot of
promise for convection-diffusion type problems including robustness in the
face of singularly perturbed problems.

\section{DPG is a Minimum Residual Method}


\bibliographystyle{plain}
\bibliography{../DPG}
\end{document}


