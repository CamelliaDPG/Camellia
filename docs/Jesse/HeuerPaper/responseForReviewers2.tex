%\documentclass[11pt,onecolumn]{scrartcl}
\documentclass{letter}
\usepackage[utf8]{inputenc}
\usepackage{amsmath,amssymb,amsfonts,mathrsfs,amsthm}
\usepackage[top=2cm,bottom=3cm,left=2.5cm,right=2cm]{geometry}
\usepackage{amssymb}
\usepackage{listings}
\usepackage{array}
\usepackage{mathtools}
\usepackage{dsfont}
\usepackage{graphicx}
\usepackage{pdfpages}
\usepackage[textsize=footnotesize,color=green]{todonotes}
\usepackage{algorithm, algorithmic}
\usepackage{array}
\usepackage{bm}
\usepackage{letterbib}
\usepackage{tikz}
%\usepackage{subfigure}
\usepackage[normalem]{ulem}

\newcommand{\bs}[1]{\boldsymbol{#1}}
\DeclareMathOperator{\diag}{diag}

\newcommand{\equaldef}{\stackrel{\mathrm{def}}{=}}

\newcommand{\tablab}[1]{\label{tab:#1}}
\newcommand{\tabref}[1]{Table~\ref{tab:#1}}

\newcommand{\theolab}[1]{\label{theo:#1}}
\newcommand{\theoref}[1]{\ref{theo:#1}}
\newcommand{\eqnlab}[1]{\label{eq:#1}}
\newcommand{\eqnref}[1]{\eqref{eq:#1}}
\newcommand{\seclab}[1]{\label{sec:#1}}
\newcommand{\secref}[1]{\ref{sec:#1}}
\newcommand{\lemlab}[1]{\label{lem:#1}}
\newcommand{\lemref}[1]{\ref{lem:#1}}

\newcommand{\mb}[1]{\mathbf{#1}}
\newcommand{\mbb}[1]{\mathbb{#1}}
\newcommand{\mc}[1]{\mathcal{#1}}
\newcommand{\nor}[1]{\left\| #1 \right\|}
\newcommand{\snor}[1]{\left| #1 \right|}
\newcommand{\LRp}[1]{\left( #1 \right)}
\newcommand{\LRs}[1]{\left[ #1 \right]}
\newcommand{\LRa}[1]{\left\langle #1 \right\rangle}
\newcommand{\LRc}[1]{\left\{ #1 \right\}}
\newcommand{\tanbui}[2]{\textcolor{blue}{\sout{#1}} \textcolor{red}{#2}}
\newcommand{\Grad} {\ensuremath{\nabla}}
\newcommand{\Div} {\ensuremath{\nabla\cdot}}
\newcommand{\Nel} {\ensuremath{{N^\text{el}}}}
\newcommand{\jump}[1] {\ensuremath{\LRs{\![#1]\!}}}
\newcommand{\uh}{\widehat{u}}
\newcommand{\fnh}{\widehat{f}_n}
\renewcommand{\L}{L^2\LRp{\Omega}}
\newcommand{\pO}{\partial\Omega}
\newcommand{\Gh}{\Gamma_h}
\newcommand{\Gm}{\Gamma_{-}}
\newcommand{\Gp}{\Gamma_{+}}
\newcommand{\Go}{\Gamma_0}
\newcommand{\Oh}{\Omega_h}

\newcommand{\eval}[2][\right]{\relax
  \ifx#1\right\relax \left.\fi#2#1\rvert}

\def\etal{{\it et al.~}}

\newcommand{\vect}[1]{\ensuremath\boldsymbol{#1}}
\newcommand{\tensor}[1]{\underline{\vect{#1}}}
\newcommand{\del}{\Delta}
\newcommand{\grad}{\nabla}
\newcommand{\curl}{\grad \times}
\renewcommand{\div}{\grad \cdot}
\newcommand{\ip}[1]{\left\langle #1 \right\rangle}
\newcommand{\eip}[1]{a\left( #1 \right)}
\newcommand{\pd}[2]{\frac{\partial#1}{\partial#2}}
\newcommand{\pdd}[2]{\frac{\partial^2#1}{\partial#2^2}}

\newcommand{\circone}{\ding{192}}
\newcommand{\circtwo}{\ding{193}}
\newcommand{\circthree}{\ding{194}}
\newcommand{\circfour}{\ding{195}}
\newcommand{\circfive}{\ding{196}}

\def\arr#1#2#3#4{\left[
\begin{array}{cc}
#1 & #2\\
#3 & #4\\
\end{array}
\right]}
\def\vecttwo#1#2{\left[
\begin{array}{c}
#1\\
#2\\
\end{array}
\right]}
\def\vectthree#1#2#3{\left[
\begin{array}{c}
#1\\
#2\\
#3\\
\end{array}
\right]}
\def\vectfour#1#2#3#4{\left[
\begin{array}{c}
#1\\
#2\\
#3\\
#4\\
\end{array}
\right]}

\newtheorem{proposition}{Proposition}
\newtheorem{corollary}{Corollary}
\newtheorem{theorem}{Theorem}
\newtheorem{lemma}{Lemma}

\newcommand{\G} {\Gamma}
\newcommand{\Gin} {\Gamma_{in}}
\newcommand{\Gout} {\Gamma_{out}}

\signature{Jesse Chan, Norbert Heuer, Tan Bui-Thanh, and Leszek Demkowicz}
\address{201 East 24th St, Stop C0200\\
Austin, Texas 78712-1229}
\begin{document}
\begin{letter}{Dr Alexey Chernov\\CAMWA Guest Editor\\Special Issue: HONAPDE 2012}

%\tableofcontents
%\maketitle

\opening{Dear Dr Chernov,}

The authors are very grateful for the feedback, comments and suggestions provided by both reviewers in this revision.  Please accept the attached second revised version of our paper. 

\textbf{Reviewer 2}

\begin{enumerate}
\item \textcolor{red}{p. 4, footnote: given that the authors are in a Hilbert space setting, the authors probably mean $\nor{v}_{V(\Oh)}^2 = \sum_{K\in \Oh} \nor{v}_{V(K)}^2$.}  Corrected -- the authors thank the reviewer for noting this.
\item \textcolor{red}{Section 1.3: the claim about the relation between the norms $\nor{\cdot}_U$ and $\nor{\cdot}_{V,U}$ is not completely clear to the referee--the authors are encouraged to add a few clarifying words:}
\begin{enumerate}
\item \textcolor{red}{please give a precise reference to a theorem in [5].} Clarified.  The theorem is specified to be Lemma 2.5 in \cite{Bui-ThanhDemkowiczGhattas11a}.  
\item \textcolor{red}{please be more specific as to what ``definite" means for the bilinear form.} Clarified.  We have added a footnote clarifying the ``definiteness" of the bilinear form as meaning 
\[
b(u,v) = 0, \forall v\in V \Rightarrow u = 0
\]
and
\[
b(u,v)=0, \forall u\in U \Rightarrow v = 0,
\]
 which imply injectivity of the bilinear operator $B$ and its transpose $B'$, defined such that $\LRa{Bu,v} = \LRa{u,B'v}.$  These conditions in turn guarantee (through the Banach closed range theorem) solvability of the variational problem.
\item \textcolor{red}{the discussion in Section $1.3$ suggests that for any norm $\nor{\cdot}_U$ one has an induced norm $\nor{\cdot}_{V,U}$ on $V$ . The referee doesn't see this. For example, taking $U = V = H_0^1(\Omega)$ and $b(u, v) \coloneqq \LRp{\grad u,\grad v}_{L^2(\Omega)}$ but equipping $U$ with $\nor{\cdot}_U \coloneqq \nor{\cdot}_{L^2(\Omega)}$ does not lead to a norm on $V$ via $\nor{\cdot}_{V,U}$ , since for smooth $v$ one obtains $\nor{v}_{V,U} = \nor{\del v}_{L^2(\Omega)}$.}  
\begin{enumerate}
\item At a very broad level, from Lemma 2.5 in \cite{Bui-ThanhDemkowiczGhattas11a}, the duality between trial and test norms is a consequence of $M/\gamma = 1$, where $M$ is the continuity constant such that $b(u,v)\leq M \nor{u}_U\nor{v}_V$, and $\gamma$ is the inf-sup/boundedness below constant such that 
\[
0<\gamma \leq \inf_{u\in U}\sup_{v\in V}\frac{b(u,v)}{\nor{u}_U\nor{v}_V}.
\]
If the problem is posed such that $U = H_0^1$ but $\nor{\cdot}_U = \nor{\cdot}_{\L}$, then this problem is ill-posed and we no longer have duality between trial and test norms.  

\item The reason for the loss of well-posedness if we choose $U = H_0^1$ and $\nor{\cdot}_U = \nor{\cdot}_{\L}$ is that the space $U = H_0^1$ is not complete under the norm $\nor{\cdot}_U = \nor{\cdot}_{\L}$.  In other words, the space $U$ is smaller than the $\L$ norm would imply - a weaker norm $\nor{\cdot}_U$ implies a large space $U$.  Under the perspective that the spaces $U$ and $V$ are defined by their respective norms, if we set $\nor{\cdot}_U = \nor{\cdot}_{\L}$, this would imply $U = \L$. Since the definitions of the spaces $U$ and $V$ are incorporated in the definition of definiteness of $b(u,v)$, we would lose definiteness of the bilinear form as well, a prerequisite for Lemma 2.5.  
\item We note that it is still possible to specify $\nor{\cdot}_U = \nor{\cdot}_{\L}$.  If we do not require $U = V = H_0^1$, and simply specify $\nor{\cdot}_U = \nor{\cdot}_{\L}$, then the space induced by $\nor{\cdot}_U$ is $U = \L$.  If, then, we rewrite the bilinear form such that $b(u,v) = -\LRp{u,\del v}_{\L}$ (to satisfy regularity constraints), then we do indeed induce the norm $\nor{v}_V \coloneqq \nor{\del v}_{\L}$, which then induces the definition of the test space $V = \LRc{v\in \L, \del v\in \L}$.  Under these conditions, we can show that both well-posedness and duality of norms hold.  

\end{enumerate}
There are two main responses to this question - the first is that the definition of a space cannot be divorced from the definition of the norm on the space (by choosing a norm on $U$, we are simultaneously choosing the space $U$).  The second Lemma 2.5 in \cite{Bui-ThanhDemkowiczGhattas11a} gives that three conditions are equivalent - 
\begin{itemize}
\item $\frac{M}{\gamma} = 1$
\item $\nor{\cdot}_U = \sup_{v\in V}\frac{b(u,v)}{\nor{v}_V}$
\item $\nor{\cdot}_{V,U} = \sup_{u\in U}\frac{b(u,v)}{\nor{u}_U}$.
\end{itemize}
These three conditions are all related to the inf-sup condition.  The starting point of DPG is to require from the start that $\frac{M}{\gamma} = 1$.  By specifying a test norm, the trial norm  (and consequentially the space $U$) are defined such that we can guarantee that $\frac{M}{\gamma} = 1$ by construction.  

\end{enumerate}
\end{enumerate}
\closing{Best regards}

\bibliographystyle{plain}
\bibliography{paper}

\end{letter}
\end{document}











