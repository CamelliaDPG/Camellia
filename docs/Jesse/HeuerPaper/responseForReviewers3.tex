%\documentclass[11pt,onecolumn]{scrartcl}
\documentclass{letter}
\usepackage[utf8]{inputenc}
\usepackage{amsmath,amssymb,amsfonts,mathrsfs,amsthm}
\usepackage[top=2cm,bottom=3cm,left=2.5cm,right=2cm]{geometry}
\usepackage{amssymb}
\usepackage{listings}
\usepackage{array}
\usepackage{mathtools}
\usepackage{dsfont}
\usepackage{graphicx}
\usepackage{pdfpages}
\usepackage[textsize=footnotesize,color=green]{todonotes}
\usepackage{algorithm, algorithmic}
\usepackage{array}
\usepackage{bm}
\usepackage{letterbib}
\usepackage{tikz}
%\usepackage{subfigure}
\usepackage[normalem]{ulem}

\newcommand{\bs}[1]{\boldsymbol{#1}}
\DeclareMathOperator{\diag}{diag}

\newcommand{\equaldef}{\stackrel{\mathrm{def}}{=}}

\newcommand{\tablab}[1]{\label{tab:#1}}
\newcommand{\tabref}[1]{Table~\ref{tab:#1}}

\newcommand{\theolab}[1]{\label{theo:#1}}
\newcommand{\theoref}[1]{\ref{theo:#1}}
\newcommand{\eqnlab}[1]{\label{eq:#1}}
\newcommand{\eqnref}[1]{\eqref{eq:#1}}
\newcommand{\seclab}[1]{\label{sec:#1}}
\newcommand{\secref}[1]{\ref{sec:#1}}
\newcommand{\lemlab}[1]{\label{lem:#1}}
\newcommand{\lemref}[1]{\ref{lem:#1}}

\newcommand{\mb}[1]{\mathbf{#1}}
\newcommand{\mbb}[1]{\mathbb{#1}}
\newcommand{\mc}[1]{\mathcal{#1}}
\newcommand{\nor}[1]{\left\| #1 \right\|}
\newcommand{\snor}[1]{\left| #1 \right|}
\newcommand{\LRp}[1]{\left( #1 \right)}
\newcommand{\LRs}[1]{\left[ #1 \right]}
\newcommand{\LRa}[1]{\left\langle #1 \right\rangle}
\newcommand{\LRc}[1]{\left\{ #1 \right\}}
\newcommand{\tanbui}[2]{\textcolor{blue}{\sout{#1}} \textcolor{red}{#2}}
\newcommand{\Grad} {\ensuremath{\nabla}}
\newcommand{\Div} {\ensuremath{\nabla\cdot}}
\newcommand{\Nel} {\ensuremath{{N^\text{el}}}}
\newcommand{\jump}[1] {\ensuremath{\LRs{\![#1]\!}}}
\newcommand{\uh}{\widehat{u}}
\newcommand{\fnh}{\widehat{f}_n}
\renewcommand{\L}{L^2\LRp{\Omega}}
\newcommand{\pO}{\partial\Omega}
\newcommand{\Gh}{\Gamma_h}
\newcommand{\Gm}{\Gamma_{-}}
\newcommand{\Gp}{\Gamma_{+}}
\newcommand{\Go}{\Gamma_0}
\newcommand{\Oh}{\Omega_h}

\newcommand{\eval}[2][\right]{\relax
  \ifx#1\right\relax \left.\fi#2#1\rvert}

\def\etal{{\it et al.~}}

\newcommand{\vect}[1]{\ensuremath\boldsymbol{#1}}
\newcommand{\tensor}[1]{\underline{\vect{#1}}}
\newcommand{\del}{\Delta}
\newcommand{\grad}{\nabla}
\newcommand{\curl}{\grad \times}
\renewcommand{\div}{\grad \cdot}
\newcommand{\ip}[1]{\left\langle #1 \right\rangle}
\newcommand{\eip}[1]{a\left( #1 \right)}
\newcommand{\pd}[2]{\frac{\partial#1}{\partial#2}}
\newcommand{\pdd}[2]{\frac{\partial^2#1}{\partial#2^2}}

\newcommand{\circone}{\ding{192}}
\newcommand{\circtwo}{\ding{193}}
\newcommand{\circthree}{\ding{194}}
\newcommand{\circfour}{\ding{195}}
\newcommand{\circfive}{\ding{196}}

\def\arr#1#2#3#4{\left[
\begin{array}{cc}
#1 & #2\\
#3 & #4\\
\end{array}
\right]}
\def\vecttwo#1#2{\left[
\begin{array}{c}
#1\\
#2\\
\end{array}
\right]}
\def\vectthree#1#2#3{\left[
\begin{array}{c}
#1\\
#2\\
#3\\
\end{array}
\right]}
\def\vectfour#1#2#3#4{\left[
\begin{array}{c}
#1\\
#2\\
#3\\
#4\\
\end{array}
\right]}

\newtheorem{proposition}{Proposition}
\newtheorem{corollary}{Corollary}
\newtheorem{theorem}{Theorem}
\newtheorem{lemma}{Lemma}

\newcommand{\G} {\Gamma}
\newcommand{\Gin} {\Gamma_{in}}
\newcommand{\Gout} {\Gamma_{out}}

\signature{Jesse Chan, Norbert Heuer, Tan Bui-Thanh, and Leszek Demkowicz}
\address{201 East 24th St, Stop C0200\\
Austin, Texas 78712-1229}
\begin{document}
\begin{letter}{Dr Alexey Chernov\\CAMWA Guest Editor\\Special Issue: HONAPDE 2012}

%\tableofcontents
%\maketitle

\opening{Dear Dr Chernov,}

The authors are very grateful for the feedback, comment and suggestion provided by reviewer 2 in this revision.  Please accept the attached third revised version of our paper. 

\textbf{Reviewer 2}

\begin{enumerate}
\item \textcolor{red}{
While the authors have answered the referee's questions in their response, the referee feels that it would be advantagenous for the readers if some of these answers were included in Section 1.3. As the authors point out in their response, the norms and the spaces are not completely independent. However, the wording in Section~{1.3} is not very clear on that point. As written, the reader may think that the {\em sets/linear spaces} $U$ and $V$ are given and he is free to choose {\em any} norm $\|\cdot\|_U$ on $U$ (as in my last example with $U = H^1_0$ equipped with the $L^2$-norm). However, as pointed out by the authors in their response, this is not what is meant. Likewise, the wording
\begin{quote}
In particular, given two {\em arbitrary} norms $\|\cdot\|_{U,1}$,
$\|\cdot\|_{U,2}$ on $U$ such that $\|\cdot\|_{U,1} \leq c \|\cdot\|_{U,2}$
\end{quote}
suggests more freedom in the choice of the norm than is intended by the authors.
.}  The authors agree, and have added an extra sentence and footnote on page 5 to caution readers against thinking so.  We have included specifically the example that the reviewer had mentioned, where it is shown that, for the standard Laplace problem with trial and test spaces in $H^1_0$, choosing for a trial norm $\nor{u}_{\L}$ (under which $H^1_0$ is not complete) leads to a test norm that is not well defined for insufficiently smooth $v\in H^1_0$.  We thank the reviewer for providing this example, which we believe illustrates the need to constrain choices of norms on test/trial spaces in a familiar and clear way to readers.  
\end{enumerate}
\closing{Best regards}

\bibliographystyle{plain}
\bibliography{paper}

\end{letter}
\end{document}











