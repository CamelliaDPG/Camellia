The goal of this work has been to explore the behavior of the Discontinuous Petrov-Galerkin (DPG) method as a method for the discretization and solution of convection-dominated diffusion problems, and produce both theory and numerical results using this method applied to model problems in this area.  We begin in Chapter~\secref{chap:intro} by describing convection-diffusion problems in fluid dynamics and the issues faced by naive methods in the convective limit, with particular emphasis on the \textit{robustness} of the method, and give a brief survey of numerical methods tailored for convection-dominated diffusion problems.  In Chapter~\secref{chap:problemRange}, we introduce the range of problems we are interested in addressing.  Specifically, we are interested in the convection-diffusion class of singular perturbation problems in computational fluid dynamics, and discuss the compressible Navier-Stokes equations, as well as the simpler model problems of Burgers' equation and the linear convection-diffusion equation, upon which we develop our numerical method.

In Chapter~\secref{chap:introDPG}, we introduce the DPG method for linear problems.  The concept of problem-dependent optimal test functions is derived through equivalence with the minimization of a specific residual, and discontinuous test functions are introduced in order to localize the determination of such optimal test functions to a single element.  The approximation of such test functions using a high order spectral method is discussed, and conclude the chapter by introducing the ultra-weak variational formulation with which the concept of optimal test functions is paired and its corresponding energy spaces.  

In Chapter~\secref{chap:graphNorm}, we show how DPG's locally constructed test space can be interpreted as a non-conforming approximation of a weakly-conforming global test space under the ultra-weak formulation.  Furthermore, the field solutions and trace solutions on the boundary $\Gamma$ are shown to depend \textit{only} upon properties of the non-conforming global test space; thus, when considering approximation error in test functions, resolution of global approximation error (as opposed to local approximation error) can be sufficient to produce a stable method.  Additionally, global properties of test spaces are given under which the ultra-weak formulation delivers the best $L^2$-approximation to the exact solution.  A connection is made to DPG through the graph test norm, which can be viewed as a regularization of the graph seminorm through the addition of an $L^2$ term of magnitude $\delta$. As the regularization parameter $\delta \rightarrow 0$, DPG under this version of the graph test norm is shown to converge to a weakly-conforming approximation to the $L^2$-optimal test space.  Finally, viewing DPG as an approximation to globally optimal test functions allows the construction of test spaces that focus on resolution of global features as opposed to local features.  We illustrate this with the convection-diffusion problem, where we show that it is possible to restore robustness with respect to the diffusion parameter $\epsilon$ by neglecting the resolution of boundary layers on the element-local level and focusing on the resolution of global boundary layers in optimal test spaces.  

In Chapter~\secref{chap:robustDPG}, we presented the analysis of a non-canonical test norm and its corresponding DPG energy norm for the convection-diffusion equation in the small-diffusion limit for solutions with strong boundary layers. Additionally, we have introduced a non-standard inflow boundary condition, and have explored the difference between between this and the standard Dirichlet inflow boundary condition. Both a definition and proof of robustness are given, and approximation of test functions is addressed.  Numerical results are presented in order to verify the results derived in this paper. However, at least for our model problem, numerical experiments appear to demonstrate results that are stronger than our proofs indicate, delivering solutions for $u$ and $\sigma$ that are extremely close to their best $L^2$ projections.   Finally, the test norm is modified to address problems with singularities.  A model problem using Laplace's equation is formulated to illustrate the presence of singular solutions to the convection-diffusion equation, and difficulties in control of singularities under the previously developed test norm are demonstrated.  Numerical experiments demonstrate that the new test norm resolves previous issues, and is effective in controlling singularities in solutions of the convection-diffusion equation.  

In the final chapter Chapter~\secref{chap:nonlinDPG}, we extend the methodology for linear problems to two model nonlinear problems.  We describe several common methods for the solution of nonlinear equations, and describe the application of the DPG method to each of them.   We demonstrate the effectiveness of DPG for nonlinear problems on a model Burgers problem with a shock solution, and then apply the DPG method to solving two model problems in supersonic/hypersonic compressible flow under different Reynolds numbers.  In particular, we demonstrate for both the flat plate and compression ramp problem in supersonic/hypersonic in compressible flow that the DPG method is able to begin from a highly underresolved meshes (two elements for the Carter plate problem, and 12 elements for the Holden ramp problem), and through automatic adaptivity, is able to resolve physical features of the solution without the aid of artificial diffusivity or shock capturing terms.  We believe this indicates both the robustness of the method on coarse grids and the effectiveness of the DPG error indicator for adaptive refinement.  

In conclusion, we have examined carefully the application of the DPG method to linear convection-dominated diffusion problem, where Galerkin test functions are computed automatically based on a choice of basis functions and the variational formulation.  We have introduced a novel variational formulation -- the ``ultra-weak'' variational formulation -- and have analyzed the nature of test functions resulting from a ``canonical'' choice of test norm.  After showing that such test functions display strong boundary layers, we concluded that resolution of the such test functions was infeasible, and developed a version of the DPG method for linear convection-dominated diffusion problems whose behavior does not degenerate as $\epsilon \rightarrow 0$.  The end goal of such an analysis was to present a method which could adaptively solve a heavily convection-dominated diffusion problem despite beginning with a highly under-resolved initial mesh.  We extrapolated such a method to a nonlinear Burgers equation and two model problems in viscous compressible flow and demonstrated its usefulness by using an automatic adaptivity scheme to fully capture features of the flow, starting with a mesh requiring no prior knowledge of the solution or physics of the simulation.

\section{Accomplishments}

In the theoretical scope of this dissertation, I have developed and proven the robustness of a Discontinuous Petrov-Galerkin method for convection-diffusion problems.  In particular, I have introduced an alternative inflow boundary condition and demonstrated its regularizing affect on the adjoint problem, allowing for the use of a stronger test norm.  Additionally, I have developed theory detailing the global nature of the DPG test space, and have shown that, for a specific series of test norms, the global DPG test space converges to a weakly-conforming approximation of the global test space under which the ultra-weak variational formulation yields the $L^2$-best approximation.  Finally, I have extended the DPG framework to nonlinear problems, demonstrating the equivalence of the DPG method to a Gauss-Newton minimization scheme for the nonlinear residual.  

In the numerical and computational scope of this dissertation, I have confirmed numerically the robustness of the DPG method for convection-diffusion problems in the convective limit under arbitrary high-order adaptive meshes.  I have implemented an anisotropic refinement scheme to more effectively capture lower-dimensional behavior of solutions of convection-diffusion problems, such as boundary layers.  Finally, I have contributed to the development of the parallel $hp$-adaptive DPG codebase Camellia\cite{Camellia}, under which the results in this dissertation were produced.  

Finally, this dissertations includes the application of the DPG method to several model convection-diffusion problems.  Convergence of the method is demonstrated under an exact solution to the scalar convection-diffusion problem, and the method is extrapolated to a nonlinear viscous Burgers' equation.  Finally, the DPG method is extrapolated to systems of equations and used to solve the flat plate and compression ramp problems in supersonic/hypersonic compressible flow.

\section{Future work}

As is the case with any research, much work remains to be done.  We outline here several areas of work which we hope to pursue in the future.
\begin{itemize}
\item \textbf{Nonlinear DPG --} as described in Section~\secref{sec:nonlinDPG}, there is a natural Hessian-based version of the DPG method which provides a second-order approximation to the nonlinear equation instead of the first-order one afforded by Newton-Raphson linearization.  Unfortunately, under this Hessian-based version of DPG, the stiffness matrix may no longer be positive definite, which can lead to non-descent search directions.  We hope to avoid such issues through the use of Newton-CG methods\cite{Nocedal2006NO}, which avoid negative search directions by terminating the CG iteration in the presence of negative curvature.  
\item \textbf{Navier-Stokes --} We have chosen the classical variables in which to cast the compressible Navier-Stokes equations; however, investigation of alternative sets of variables may have merit, as different choices of variables yield differing linearizations with their own advantages (for example, all derivatives in time are linear with respect to the momentum variables, and the entropy variables of Hughes both symmetrize the Navier-Stokes equations and yield solutions obeying second law of thermodynamics for standard $H^1$ formulations \cite{Hughes1986223}).  

We also hope to investigate artificial viscosity methods as regularization for problems in viscous compressible flow.  We present an analysis of the 1D Burgers' equation demonstrating that the exact solutions under Newton linearization contain large oscillations.  While these oscillations are not the result of the stability of the spatial discretization, their presence can cause density and temperature to become non-physically negative, which can stall the convergence of the nonlinear solver.  We hope to investigate artificial viscosity not as a stabilization mechanism of the discrete spatial discretization, but as a regularization of the strong problem with which to suppress the presence of large oscillations in the linearized solution.  

Finally, though the method is inf-sup stable for arbitrary meshes, most of our experiments have focused on meshes of uniform $p$.  We hope to implement a true $hp$-adaptive DPG method for the compressible Navier-Stokes equations in the future.  
\item \textbf{Alternative discretizations --} Recent works (see \cite{DahmenVariationalStabilization, broersenStevenson, primalDPG, H1DPG}) have applied the same minimum residual methodology behind DPG to alternative discretizations, as well as the use of \textit{continuous} test functions.\footnote{We note that the use of continuous test functions does not imply the computation of such test functions over the entire mesh; it is shown that the minimum residual method can be formulated instead as a saddle point problem, which is equivalent to computing optimal test functions.  See \cite{DahmenVariationalStabilization, H1DPG} for more details.}  We hope to investigate the behavior of the minimum residual method under both continuous test functions and different variational formulations. 
\item \textbf{Alternative architectures --} For an efficient implementation of DPG, massively parallel low memory architectures are required.  In this work, we have focused on an MPI implementation of DPG.  However, future access to extremely large MPI-based clusters may be limited to those who can afford the cost of petascale -- namely, government-sponsored projects and large engineering companies.  We hope to experiment with the GPU implementation of DPG as a lower-cost, highly parallel HPC alternative.  
\end{itemize}